% Options for packages loaded elsewhere
\PassOptionsToPackage{unicode}{hyperref}
\PassOptionsToPackage{hyphens}{url}
%
\documentclass[
]{ctexart}
\usepackage{amsmath,amssymb}
\usepackage{lmodern}
\usepackage{ifxetex,ifluatex}
\ifnum 0\ifxetex 1\fi\ifluatex 1\fi=0 % if pdftex
  \usepackage[T1]{fontenc}
  \usepackage[utf8]{inputenc}
  \usepackage{textcomp} % provide euro and other symbols
\else % if luatex or xetex
  \usepackage{unicode-math}
  \defaultfontfeatures{Scale=MatchLowercase}
  \defaultfontfeatures[\rmfamily]{Ligatures=TeX,Scale=1}
\fi
% Use upquote if available, for straight quotes in verbatim environments
\IfFileExists{upquote.sty}{\usepackage{upquote}}{}
\IfFileExists{microtype.sty}{% use microtype if available
  \usepackage[]{microtype}
  \UseMicrotypeSet[protrusion]{basicmath} % disable protrusion for tt fonts
}{}
\makeatletter
\@ifundefined{KOMAClassName}{% if non-KOMA class
  \IfFileExists{parskip.sty}{%
    \usepackage{parskip}
  }{% else
    \setlength{\parindent}{0pt}
    \setlength{\parskip}{6pt plus 2pt minus 1pt}}
}{% if KOMA class
  \KOMAoptions{parskip=half}}
\makeatother
\usepackage{xcolor}
\IfFileExists{xurl.sty}{\usepackage{xurl}}{} % add URL line breaks if available
\IfFileExists{bookmark.sty}{\usepackage{bookmark}}{\usepackage{hyperref}}
\hypersetup{
  pdftitle={准备工作及一些操作流程},
  hidelinks,
  pdfcreator={LaTeX via pandoc}}
\urlstyle{same} % disable monospaced font for URLs
\usepackage{graphicx}
\makeatletter
\def\maxwidth{\ifdim\Gin@nat@width>\linewidth\linewidth\else\Gin@nat@width\fi}
\def\maxheight{\ifdim\Gin@nat@height>\textheight\textheight\else\Gin@nat@height\fi}
\makeatother
% Scale images if necessary, so that they will not overflow the page
% margins by default, and it is still possible to overwrite the defaults
% using explicit options in \includegraphics[width, height, ...]{}
\setkeys{Gin}{width=\maxwidth,height=\maxheight,keepaspectratio}
% Set default figure placement to htbp
\makeatletter
\def\fps@figure{htbp}
\makeatother
\setlength{\emergencystretch}{3em} % prevent overfull lines
\providecommand{\tightlist}{%
  \setlength{\itemsep}{0pt}\setlength{\parskip}{0pt}}
\setcounter{secnumdepth}{-\maxdimen} % remove section numbering
\ifluatex
  \usepackage{selnolig}  % disable illegal ligatures
\fi

\title{准备工作及一些操作流程}
\author{}
\date{\vspace{-2.5em}2021/8/29}

\begin{document}
\maketitle

\hypertarget{ux51c6ux5907ux5de5ux4f5c}{%
\subsection{准备工作}\label{ux51c6ux5907ux5de5ux4f5c}}

1.注册一个\href{https://github.com}{\textbf{GitHub账户}}\\
2.安装或者更新\textbf{R}和\textbf{RStudio}\\
3.安装\href{https://gitforwindows.org/}{\textbf{Git(windows)}}\\
4.已经在Git上介绍了自己\\
git config --global user.name `leimingri'\\
git config --global user.email
`\href{mailto:lmr18845128812@163.com}{\nolinkurl{lmr18845128812@163.com}}'\\
git config --global --list\\
替换您的名称和与GitHub帐户相关的电子邮件\\
5.已确认可以从命令行对GitHub推/拉

\hypertarget{ux8fdeux63a5git-githubrstudio}{%
\subsection{连接Git
GitHub,RStudio}\label{ux8fdeux63a5git-githubrstudio}}

首先,在GitHub上面创建一个项目repository,然后将项目地址克隆到RStudio中,接着进行本地的更改、保存及提交,将新增更改的内容保存到GitHub该项目存储库中,下面将运用图示法进行详细的流程介绍。\\
1.连接RStudio到Git和GitHub
在GitHub上创建存储库(项目),然后通过RStudio将新的GitHub存储库克隆到您的计算机上。在RStudio中,\emph{File
\textgreater{} New Project \textgreater{} Version Control \textgreater{}
Git}:\\
\includegraphics{http://r.photo.store.qq.com/psc?/V54AC60s2AQkQe24IJrU0a9knd0j1QQg/45NBuzDIW489QBoVep5mcWqp6iCnBt6LgMYnnO6B5HpXjXKlbYxFk46xki6v1zZcEaLKDpUiI.MLcBVFuPNGdKY.JLtle6Q57.*tknTqg3k!/r}\\
\includegraphics{http://r.photo.store.qq.com/psc?/V54AC60s2AQkQe24IJrU0a9knd0j1QQg/45NBuzDIW489QBoVep5mcT7P9tdwXVcQslYJNuCoe7W6hmCqubVl3HuUivpwPir.KRmHsZwkPDJyMmRkW4WZ6C2jNxt*phmYwxIP7IVIALo!/r}\\
\includegraphics{http://r.photo.store.qq.com/psc?/V54AC60s2AQkQe24IJrU0a9knd0j1QQg/45NBuzDIW489QBoVep5mcT7P9tdwXVcQslYJNuCoe7VnCuhp9oP3UJXscyIvhykoRRAjVXj5hwQz1nFPXdgwFJhCCezbOKtjKDMgqj0vSv0!/r}
\includegraphics{http://r.photo.store.qq.com/psc?/V54AC60s2AQkQe24IJrU0a9knd0j1QQg/45NBuzDIW489QBoVep5mcT7P9tdwXVcQslYJNuCoe7VyroulROJQ3b42cW5GC0dZCEBNkSoiv4IA*whdlrb834MPsCW0Jk7Y1GrpFMW*ZFM!/r}

2.进行本地更改,保存、提交:下载我们在GitHub上创建的README.md文件,在RStudio的文件浏览器窗格中查看README.md文件,从RStudio中修改文件。\\
\includegraphics{ttp://r.photo.store.qq.com/psc?/V54AC60s2AQkQe24IJrU0a9knd0j1QQg/45NBuzDIW489QBoVep5mcT7P9tdwXVcQslYJNuCoe7XxWjr.9heo*14lh0n7j..AY*WsAR3aJ763ojGHfyK0m09NzLViBpZiLcDkw99sjCQ!/r}
3.保存更改,将您在本地进行的更改在线推送到GitHub,操作如下:

4.最后,确认传播到GitHub远程存储器的本地更改。\\
注:\href{http://rmarkdown.rstudio.com}{R Markdown官网地址}\\
以上内容参考文献\href{https://happygitwithr.com/}{happy-git-with-R}

\hypertarget{ux9047ux5230ux7684ux95eeux9898}{%
\subsection{遇到的问题}\label{ux9047ux5230ux7684ux95eeux9898}}

1.在RStudio中直接导入GitHub中项目的URL与在RStudio中创建项目再将RStudio中的项目导入GitHub中,两者区别是什么?(将RStudio中的项目导入GitHub中具体操作是怎样的,应该通过在RStudio中键入GitHub账号邮箱还是在GitHub中直接导入项目所在位置呢)\\
2.rmarkdown中采取``-''没有生成无序列表,是什么原因呢?\\
3.怎样将自己操作步骤的截图图片放入到rmarkdown文档中?\\
4.导入GitHub中的URL时,出现`\ldots Connection was reset, errno
10054'错误时,解决办法:\emph{git config --global http.sslVerify
``false''}即解除ssl验证,再次git\\
5.设置图片的大小和位置问题\\
6.为什么生成的pdf里面没有图片,而html里面可以显示出图片?

\end{document}
